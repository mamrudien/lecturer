\begin{problem}{Dosen Matematika}{standard input}{standard output}{1 second}{256 megabytes}

Pak Blangkon adalah seorang dosen matematika di Universitas Gudang Mantu yang menyukai bilangan bulat positif $N$.

Suatu hari, ketika sedang mengajar, ia secara mengejutkan mengajukan pertanyaan pada mahasiswa-mahasiswinya, 
\begin{itemize}
\item Misalkan $a$ dan $b$ masing-masing adalah suatu bilangan bulat nonnegatif kurang dari $N$ di mana $a \neq b$,
\item Kemudian misalkan $S$ sebuah himpunan di mana $a,b\in{S}$,
\item Untuk sembarang $x,y \in S$ di mana $x \neq y$, maka $z\in S$ di mana $z=(x+y)\operatorname{mod}N$.
\item Berapa banyak anggota $S$?
\end{itemize}

Cobalah jawab pertanyaan Pak Blangkon tersebut!

\InputFile
Baris pertama berisi tepat satu buah bilangan bulat positif $T$ ($1 \le T \le 1000$) yang menyatakan banyaknya kasus uji.

Kemudian $T$ baris berikutnya masing-masing berisi tepat tiga buah bilangan bulat $N$, $a$, dan $b$ ($2\leq{N}\leq{10^{18}}, 0 \le a,b<{N},a\neq{b}$) yang dipisahkan oleh spasi.

\OutputFile
Bilangan bulat non-negatif yang menyatakan banyak anggota $S$ untuk setiap kasus uji.

\Example

\begin{example}
\exmpfile{example.01}{example.01.a}%
\end{example}

\end{problem}

